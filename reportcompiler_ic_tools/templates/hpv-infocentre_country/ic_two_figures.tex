\begin{figure}[h]
  \caption{\VAR{ctx.caption}}
  \centering
    \fbox{\includegraphics[width=\VAR{ctx.image_width | default('1')}\textwidth]{\VAR{ctx.image_path_1 | escape_path}}
	\includegraphics[width=\VAR{ctx.image_width | default('1')}\textwidth]{\VAR{ctx.image_path_2 | escape_path}}}
\end{figure}

\BLOCK{include '/hpv-infocentre/ic_references_no_markers.tex'}